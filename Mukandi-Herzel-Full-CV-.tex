% Options for packages loaded elsewhere
\PassOptionsToPackage{unicode}{hyperref}
\PassOptionsToPackage{hyphens}{url}
%
\documentclass[
  10pt,
]{article}
\usepackage{amsmath,amssymb}
\usepackage{setspace}
\usepackage{iftex}
\ifPDFTeX
  \usepackage[T1]{fontenc}
  \usepackage[utf8]{inputenc}
  \usepackage{textcomp} % provide euro and other symbols
\else % if luatex or xetex
  \usepackage{unicode-math} % this also loads fontspec
  \defaultfontfeatures{Scale=MatchLowercase}
  \defaultfontfeatures[\rmfamily]{Ligatures=TeX,Scale=1}
\fi
\usepackage{lmodern}
\ifPDFTeX\else
  % xetex/luatex font selection
\fi
% Use upquote if available, for straight quotes in verbatim environments
\IfFileExists{upquote.sty}{\usepackage{upquote}}{}
\IfFileExists{microtype.sty}{% use microtype if available
  \usepackage[]{microtype}
  \UseMicrotypeSet[protrusion]{basicmath} % disable protrusion for tt fonts
}{}
\makeatletter
\@ifundefined{KOMAClassName}{% if non-KOMA class
  \IfFileExists{parskip.sty}{%
    \usepackage{parskip}
  }{% else
    \setlength{\parindent}{0pt}
    \setlength{\parskip}{6pt plus 2pt minus 1pt}}
}{% if KOMA class
  \KOMAoptions{parskip=half}}
\makeatother
\usepackage{xcolor}
\usepackage[top=0.3in, bottom=0.3in, left=0.4in, right=0.4in]{geometry}
\usepackage{graphicx}
\makeatletter
\newsavebox\pandoc@box
\newcommand*\pandocbounded[1]{% scales image to fit in text height/width
  \sbox\pandoc@box{#1}%
  \Gscale@div\@tempa{\textheight}{\dimexpr\ht\pandoc@box+\dp\pandoc@box\relax}%
  \Gscale@div\@tempb{\linewidth}{\wd\pandoc@box}%
  \ifdim\@tempb\p@<\@tempa\p@\let\@tempa\@tempb\fi% select the smaller of both
  \ifdim\@tempa\p@<\p@\scalebox{\@tempa}{\usebox\pandoc@box}%
  \else\usebox{\pandoc@box}%
  \fi%
}
% Set default figure placement to htbp
\def\fps@figure{htbp}
\makeatother
\setlength{\emergencystretch}{3em} % prevent overfull lines
\providecommand{\tightlist}{%
  \setlength{\itemsep}{0pt}\setlength{\parskip}{0pt}}
\setcounter{secnumdepth}{-\maxdimen} % remove section numbering
\usepackage{enumitem}
\setlist[itemize]{itemsep=0.2em, topsep=0.2em, parsep=0em, partopsep=0em, leftmargin=1.5em}
\usepackage{titlesec}
\titlespacing\section{0pt}{0.5em}{0.1em}
\titleformat{\section}{\normalfont\normalsize\bfseries}{\thesection}{0em}{}
\titlespacing\subsection{0pt}{0.3em}{0.1em}
\titleformat{\subsection}{\normalfont\normalsize\bfseries}{\thesubsection}{0em}{}
\usepackage{xcolor}
\usepackage{hyperref}
\hypersetup{colorlinks=true, urlcolor=blue}
\usepackage{bookmark}
\IfFileExists{xurl.sty}{\usepackage{xurl}}{} % add URL line breaks if available
\urlstyle{same}
\hypersetup{
  pdftitle={Curriculum Vitae - Herzel Shingirirai Mukandi},
  hidelinks,
  pdfcreator={LaTeX via pandoc}}

\title{Curriculum Vitae - Herzel Shingirirai Mukandi}
\author{}
\date{\vspace{-2.5em}}

\begin{document}
\maketitle

\setstretch{1.25}
\vspace{-2em}

\begin{center}
{\small
Email: \textcolor{blue}{\href{mailto:mukandiherzel@gmail.com}{mukandiherzel@gmail.com}} \quad
LinkedIn: \textcolor{blue}{\href{https://www.linkedin.com/in/herzelmukandi}{linkedin.com/in/herzelmukandi}} \quad
GitHub: \textcolor{blue}{\href{https://github.com/HerzelS}{github.com/HerzelS}}
}
\end{center}

\vspace{0.5em}

\section{ABOUT ME}\label{about-me}

\textbf{Monitoring and Evaluation} specialist with over \textbf{10 years
of experience} in designing, managing, and implementing evaluations for
national, regional and international development projects, particularly
in rural development, gender, and humanitarian contexts. Adept at
utilizing qualitative and quantitative research methodologies, including
impact evaluation, economic analysis, and risk management.

\vspace{0.5cm}

\section{SKILLS}\label{skills}

\begin{itemize}
\item
  \textbf{Data Collection}: Kobo Collect, ODK, Microsoft Forms, Google
  Forms, Survey Monkey
\item
  \textbf{Data Management, Analysis and Visualisation}: R, Python, SPSS,
  Nvivo, Microsoft PowerBI, SQL
\item
  \textbf{Communication and Knowledge Management}: Microsoft Teams,
  Zoom, SharePoint, Miro, MS Project
\item
  \textbf{AI}: Developing RAG Systems Algorithms, Script Engineering
\item
  \textbf{Machine Learning}: Neural Networks, KNN, Naive Bayes,
  Regression Models, Sector Vector Machines
\end{itemize}

\vspace{0.5cm}

\section{PROFESSIONAL WORK EXPERIENCE}\label{work-experience}

\subsection{\texorpdfstring{Associate Evaluation Officer - UNHCR
\emph{03/01/2024 -- Current} \textbar{} \emph{Geneva, Switzerland}}{Associate Evaluation Officer - UNHCR 03/01/2024 -- Current \textbar{} Geneva, Switzerland}}\label{associate-evaluation-officer---unhcr-03012024-current-geneva-switzerland}

Under the overall supervision of the Regional Senior Evaluation Officer(s), assist in managing and delivering Humanitarian Emergency, Global, Regional and Country Thematic Strategic, Project and Joint evaluations commisioned as part of the EvO rolling workplan. Additionally, am providing strategic support to evaluation Capacity Building across Souther Arica and the East Horn of Africa and the Great Lakes (EHAGL). I am also the gender and Enterprise risk management focal persons for the EvO.

\begin{itemize}
\tightlist
\item
Supporting global, regional and country-level evaluations across 30+ countries (Africa, Europe, Asia, the Middle and the Americas), ensuring alignment with the UNHCR Evaluation policy, methodologies, and normative frameworks.
\item
Providing quality assurance by reviewing evaluation deliverables enhancing evaluation quality and credibility.
\item
Guiding the Regional Bureau for EHAGL and Southern Africa RB and 16 country offices in their strategic annual and multi-year planning developing quality Results Frameworks and Theories to Change.
\item
Exploring evaluation synthesis efficiency through piloting AI-based tools for automated RAG extracted reports using Ollama and Python as well as implementing innovative scalable and reproducible Fusen packages in R.
\item
Collected and analysed data, prepared desk review notes, analysis papers, and Evaluation Office annual report sections, and disseminated key evaluation findings, best practices, and lessons learned.
\item
Applied risk management principles as the Risk Management focal Point for the Evaluation office, identified, analyzed, and recommended mitigation measures, improving risk mitigation strategies by 40\%.
\item
Contributed to internal knowledge-sharing initiatives by developing a dynamic Power BI dashboard that monitors performance for 150+ evaluative projects in MS Project.
\item
Participated in interagency collaborations including in the UNEG Human Rights and Disability Working Group.
\end{itemize}

\subsection{\texorpdfstring{Monitoring And Evaluation Officer - UN Women
\emph{01/02/2022 -- 02/01/2024} \textbar{} \emph{Jakarta, Indonesia}}{Monitoring And Evaluation Officer - UN Women 01/02/2022 -- 02/01/2024 \textbar{} Jakarta, Indonesia}}\label{monitoring-and-evaluation-officer---un-women-01022022-02012024-jakarta-indonesia}

I was responsible for Results Monitoring, Evaluation and Reporting for a regional Women, Peace, and Security program (2020-2025) implemented regionally in Southeast Asia and nationally in five countries, namely Indonesia, Thailand, Vietnam, Timore-Leste and the Philippines.

\begin{itemize}
\tightlist
\item
\textbf{Guided gender-responsive and human rights evaluation} and contributed to the ASEAN WPS project Baseline evaluation (2021-2022), Evaluability assessment (2022), project Mid-Term Review (2023) and Indonesia Country Portfolio Evaluation (2023).
\item
\textbf{Solved the problem of irreconcilable data from different tools} by \emph{developing an M and E System} with consolidated data collection tools in \emph{Microsoft Forms} that tracked indicators improving the completeness of data from 75\% to 100\%. \textbf{Eliminated the problem of double counting} by migrating the project database from \emph{MS Excel} to a \emph{PostgreSQL database}.
\item
Significantly reduced the time for collecting, collating, cleaning, analysing data and preparing activity reports by \textbf{automating reporting} using a \emph{datapipeline} that extracted post-test surveys from \emph{Microsfot Forms} into a \emph{Postrgesql} (for storage), then to \emph{R} for analaysis \textbf{automatically rendering} activity reports to \emph{MS Word} documents using \emph{R-Markdown} and radiating the results in an \emph{MS PowerBI} dahsboard for communicating progress against indicators.
\item
Provided capacity development support by training 17 staff and 15 implementing partners on monitoring and evaluation processes and tools, resulting in a 60\% increase in staff proficiency in RBM principles.
\item
Maintained the risk register and supported the drafting of annual donor and UN Women corporate reports.
\end{itemize}

\subsection{\texorpdfstring{Monitoring And Evaluation Specialist - Young
Africa International (YAI) \emph{01/11/2020 -- 31/01/2022} \textbar{}
\emph{Harare, Zimbabwe}}{Monitoring And Evaluation Specialist - Young Africa International (YAI) 01/11/2020 -- 31/01/2022 \textbar{} Harare, Zimbabwe}}\label{monitoring-and-evaluation-specialist---young-africa-international-yai-01112020-31012022-harare-zimbabwe}

Coordinated results for over 5 regional projects in 5 countries (Zimbabwe, Zambia, Mozambique, Botswana and Namibia) - portfolio value 15m USD. Donors: European Union, the Royal Embassy of the Netherlands, NUFFIC, SAS Foundation.

\begin{itemize}
\tightlist
\item
Managed two multi-partner TVET project evaluations (2022)- A utilisation-focused and an Outcomes evaluation, met critical deadlines and delivered quality evaluations on schedule and within budget.
\item
Led a Real-Time evaluation for the Cyclone Idai reconstruction and resilience building project implemented in Beira, Mozambique (2020-2021), funded by the European Union.
\item
Conducted monthly coordination meetings with 5 project team members and carried quarterly monitoring visits which improved Key Performance Indicator efficiency from 75\% to 98\%.
\item
Conducted coordination meetings, and monitoring visits, achieving a Key Performance Indicator (KPI) efficiency from 75\% to 98\%, utilizing SPSS for data analysis and Power BI for visual reporting.
\item
Improved Monitoring, Evaluation and Learning (MEL) capacities, practices and data-driven decision-making through four capacity-building trainings for 47 staff members and 23 partner organizations.
\item
Contributed to internal knowledge-sharing platforms, ensuring best practices were adopted in program implementation, leading to a 50\% increase in knowledge retention.
\end{itemize}

\subsection{\texorpdfstring{Knowledge and Humanitarian Research Manager - Childline \emph{01/02/2020 -- 31/10/2022} \textbar{} \emph{Harare, Zimbabwe}}{Knowledge and Humanitarian Research Manager - Childline 01/02/2020 -- 31/10/2022 \textbar{} Harare, Zimbabwe}}\label{knowledge-and-humanitarian-research-manager---childline-01022020-31102022-harare-zimbabwe}

Childline Zimbabwe is the biggest child protection organisation in Zimbabwe which operates a 116-call centre in Zimbabwe, reaching over 250 000 children each year. I led the Knowledge and Humanitarian Research department which had four regional departments implementing over ten projects in total.

\begin{itemize}
\tightlist
\item
Led a Rural Rapid Appraisal on Sexual Gender-Based Violence and COVID-19 emergency which placed over 500 children into places of safety in Harare, captured by local media Nehanda Radio and H-Metro.
\item
Led the organisational COVID-19 response and collected real-time data on the high rise of child abuse cases during the COVID-19 lockdown, and was interviewed by multiple media including, \href{https://www.sundaymail.co.zw/is-your-child-safe-shocking-rise-in-child-abuse-cases}{Sunday Mail}, \href{https://nehandaradio.com/2020/04/17/62-children-removed-from-the-streets/}{Nehanda Radio}, \href{https://www.sundaymail.co.zw/school-misdemeanours-tip-of-the-iceberg}{Sunday Mail}.
\item
Managed an Outcomes Project Evaluation for a re-integration project at the Tongogara Refugee Camp (2016-2020), managed stakeholder communication and expectations with the final evaluation report receiving a highly satisfactory rating.
\item
Conducted data analysis using SPSS for quantitative and content and thematic analysis for qualitative data which provided evidence-based insights for project decision-making.
\item
Enhanced indicator tracking efficiency from 74\% to 96\% in Q3 of 2020 through conducting scheduled monitoring visits and spot checks ensuring implementation of recommendations.
\item
Developed customized communication products for knowledge dissemination to partners and stakeholders, increasing stakeholder engagement by 20\%.
\end{itemize}

\subsection{\texorpdfstring{Monitoring and Evaluation Officer - Legal Resources Foundation \emph{30/06/2017 -- 30/01/2020} \textbar{} \emph{Harare, Zimbabwe}}{Monitoring and Evaluation Officer - Legal Resources Foundation 30/06/2017 -- 30/01/2020 \textbar{} Harare, Zimbabwe}}\label{monitoring-and-evaluation-officer---legal-resources-foundation-30062017-30012020-harare-zimbabwe}

Legal Resources Foundation (LRF) is a local Non-Governmental Organisational providing legal education, legal advice, legal representation, strategic litigation, and advocacy for the over 30,000 vulnerable people in Zimbabwe.

\begin{itemize}
\tightlist
\item
Designed a Results-Based M\&E framework, achieving a 95\% data verification rate, utilizing MS Excel for tracking project outcomes across 10 provinces (3 consortiums with 5 partners).
\item
Conducted baseline, process and end-line evaluations focusing on governance, human rights, and access to justice programs, producing useful and credible evaluative data.
\item
Engaged in peer reviews and quality assurance improving consortium annual reports accuracy by 40\%.
\item
Developed data collection tools including surveys, focused group discussion guides, interview guides and checklists to support routine monitoring of projects with over 95\% KPI efficiency.
\item
Conducted regular monitoring of projects, ensuring effective resource utilization and risk mitigation, reducing implementation costs by more than 15\%.
\item
Improved efficiency by developing and integrating a Data Quality Assurance Framework, developing an automated MS-Excel based database system to store and manage data across various projects.
\end{itemize}

\subsection{\texorpdfstring{Monitoring and Evaluation Advisor - Zimbabwe
National Network of People Living with HIV (ZNNP+) \emph{21/07/2016 --
29/06/2017} \textbar{} \emph{Harare, Zimbabwe}}{Monitoring and Evaluation Advisor - Zimbabwe National Network of People Living with HIV (ZNNP+) 21/07/2016 -- 29/06/2017 \textbar{} Harare, Zimbabwe}}\label{monitoring-and-evaluation-advisor---zimbabwe-national-network-of-people-living-with-hiv-znnp-21072016-29062017-harare-zimbabwe}

Zimbabwe National Network of People Living with HIV (ZNNP+) is the national network of people living with HIV in Zimbabwe providing support to and policy advocacy for over 250 000 members in the 10 provinces of Zimbabwe.

\begin{itemize}
\tightlist
\item
Collaborated with district officers to conduct 50+ monitoring visits for the Trócaire-led Comic Relief-funded project, ensuring that activities were properly implemented and increasing community participation by 30\%.
\item
Coordinated with six partners in the mid-term and end-line evaluations, and oversaw the planning, preparation, conducting, reporting and use and follow-up of the evaluation findings.
\item
Developed and maintained the organisation's monitoring and evaluation system based on DHIS 2.
\item
Represented the organisation in key networking events and various consortiums including National Sector Coordination Meetings, and review meetings with government, civil society, and other partners.
\item
Prepared highly quality reports for donors and stakeholders resulting in a 100\% timely reporting efficiency.
\end{itemize}

\subsection{\texorpdfstring{Monitoring and Evaluation Officer - St Theresa Ruvheneko \emph{01/06/2014 -- 30/06/2016} \textbar{} \emph{Gweru, Zimbabwe}}{Monitoring and Evaluation Officer - St Theresa Ruvheneko 01/06/2014 -- 30/06/2016 \textbar{} Gweru, Zimbabwe}}\label{monitoring-and-evaluation-officer---st-theresa-ruvheneko-01062014-30062016-gweru-zimbabwe}

St Theresa’s Hospital runs the Ruvheneko Centre in Chirumhanzu District, Midlands Province, Zimbabwe. It was the first hospital to provide free Ati-Retroviral Treatment in 2022 and has provided HIV-related services to over 500,000 clients across the country. 

\begin{itemize}
\tightlist
\item
Co-wrote a successful funding proposal for an Action Research Ladder to Safety project, resulting in a 2-year project funded by the Swiss Agency for Development Corporation and Voluntary Services Overseas.
\item
Collaborated in a Utilisation-Focused Evaluation of the Ladder to Safety Project with stakeholders through participatory methods such as group work and interviews.
\item
Prepared the final project report of the Ladder to Safety Project receiving high praise from the Project Steering Committee.
\end{itemize}

\vspace{0.5cm}


\section{EDUCATION AND TRAINING}\label{education-and-training}

\begin{itemize}[itemsep=0.25em, topsep=0.25em]

\item Master of Social Work, University of Zimbabwe, Harare, Zimbabwe (31/05/2017 – 30/12/2018)

\item Special Honors Degree in Monitoring and Evaluation, Lupane State University, Lupane, Zimbabwe (31/12/2019 – 29/12/2021) 
  
\item Honors Degree Social Work, University of Zimbabwe, Harare, Zimbabwe (31/07/2010 – 29/06/2014)
\end{itemize}

\vspace{0.5cm}

\section{PROFESSIONAL CERTIFICATIONS}\label{professional-certifications}

\begin{itemize}[itemsep=0.25em, topsep=0.25em]
  \item Program Management Professional (PgMP) - Project Management Institute, USA, Renewing May 2026  
  \item Project Management Professional (PMP), Project Management Institute, Renewing 5 June 2027  
  \item PRINCE2 Foundations, AXELOS Global Best Practice, No Expiry Date  
  \item PMI Agile Certified Practitioner (PMI-ACP), Project Management Institute, Renewing 10 November 2027  
  \item PMI Risk Management Professional (PMI-RMP), Project Management Institute, Renewing 19 December 2027
\end{itemize}

\vspace{0.5cm}

\section{PUBLICATIONS}\label{publications}

\begin{itemize}
\tightlist
\item
  Mukandi, H. S., Taruvinga, R., and Muzingili, T. (2020).
  \href{https://catalogue.leidenuniv.nl/permalink/31UKB_LEU/s5ab2f/alma9939992501102711}{Assessing
  the viability of vocational education as a livelihood strategy for
  persons with disabilities in Zimbabwe.}
\end{itemize}

\vspace{0.5cm}

\section{NETWORKS AND MEMBERSHIPS}\label{networks-and-memberships}

\begin{itemize}[itemsep=0.25em, topsep=0.25em]
  \item Project Management Institute (Since 01/06/2021, Pennsylvania, USA)  
  \item United Nations Evaluation Group (UNEG) (Since 01/01/2024, Geneva, Switzerland)
\end{itemize}

\vspace{0.5cm}

\section{VOLUNTEERING}\label{volunteering}

\begin{itemize}[itemsep=0.25em, topsep=0.25em]
  \item Youth and Social Impact Director, Project Management Institute Zimbabwe \textit{(01/11/2022 – 31/12/2023)}
  \item Student Mentor, Project Management Institute South Africa \textit{(01/01/2021 – 31/12/2021)}
\end{itemize}

\end{document}
